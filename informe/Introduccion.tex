\section{Introducción}

En este informe, vamos a analizar resoluciones distintas para el problema presentado a continuación.

Problema: Dada una secuencia de numeros naturales, se pide  pintar de color rojo a una secuencia estrictamente creciente y de color azul a una secuencia estrictamente decreciente
de manera tal que se minimice la cantidad de elementos sin pintar.

Por ejemplo dada la secuencia:
\begin{flushleft}
\newline
  \begin{tabular}{ | l | c | r | l | c | r | l | c | r | l |}
    \hline
1 & 2 & 2 & 1 & 1 & 2 & 3 & 3 & 2 & 1\\ \hline
  \end{tabular}
\newline
Rojo:
\newline
  \begin{tabular}{ | l | c | r |}
    \hline
    1 & 2 & 3 \\ \hline
  \end{tabular}
 \newline
  Azul:
  \newline
  \begin{tabular}{ | l | c | r |}
    \hline
    3 & 2 & 1 \\ \hline
  \end{tabular}
  \newline
Resultado = 4
\vspace*{5mm}
\newline
  \begin{tabular}{ | l | c | r | l | c | r | l |}
    \hline
7 & 2 & 4 & 5 & 5 & 3 & 9\\ \hline
  \end{tabular}
\newline
Rojo:
\newline
  \begin{tabular}{ | l | c | r | r |}
    \hline
    2 & 4 & 5 & 9\\ \hline
  \end{tabular}
 \newline
  Azul:
  \newline
  \begin{tabular}{ | l | c | r |}
    \hline
    7 & 5 & 3 \\ \hline
  \end{tabular}
  \newline
Resultado = 0

\vspace*{5mm}

Para ello resolveremos este problema de tres formas distintas:

\vspace*{5mm}
\begin{itemize}
    \item Backtracking.
    \item Backtracking con poda.
    \item Dinámica.
\end{itemize}

\end{flushleft}
