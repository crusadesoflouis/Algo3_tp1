\section{Introducción}

En este informe explicaremos el desarrollo del código realizado para resolver los problemas pertenecientes al Trabajo Práctico 1 y la justificación de la complejidad obtenida de cada ejercicio.

Los problemas serán resueltos utilizando las técnicas algorítmicas conocidas como Backtracking, programación dinámica y algoritmos golosos. 

El código está desarrollado en Java y los gráficos los generamos a partir de la pagina de internet para crear gráficos a partir de tablas de datos, https://plot.ly/ y el programa gnuplot.

Es importante destacar además que a medida que se resolvieron los problemas, las soluciones fueron probadas para verificar el funcionamiento del código que serán presentadas como tests.

Para justificar la complejidad de los algoritmos que resuelven los problemas se utiliza pseudocódigo 
%con un comentario en cada linea con la complejidad 
y luego una justificación de la complejidad del algoritmo. Por último, se realizará una experimentación para comparar el funcionamiento del algoritmo con la complejidad estimada.
