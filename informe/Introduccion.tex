\section{Introducción}

En este informe, vamos a analizar resoluciones distintas para el problema presentado a continuación.

Problema: Dada una secuencia de numeros naturales, se pide encontrar pintar de color rojo a una secuencia estrictamente creciente y de color azul a una secuencia estrictamente decreciente
de manera tal que se maximice la cantidad de elementos pintados.

Por ejemplo dada la secuencia 5 8 1 0 4 7 8  rojo = 0 4 7 8 azul = 5 8 1
1 2 3 3 2 1 1 2 3 4 4 3 2 1 = rojo 1 2 3 4 azul =  4 3 2 1

1 2 3 4 4  = 1 2 3 4,  4
1 2 3 4 4 4 = 1 2 3 4, 4



Para ello resolveremos este problema de tres formas distintas:
La primera será resuelta por medio de Backtracking,en la segunda se le agregará una poda al mismo algoritmo del ejercicio 1 y por ultimo se explicará un algoritmo de programacion dinámica.

Cada ejercicio presentará con una introducción explicación, hipotesis de complejidad, resolución y conclusion.
