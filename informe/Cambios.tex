\section{Hoja de cambios:}

\subsection{Problema 1}

\begin{itemize}

\item En la tabla de ejemplos de la sección 2.1.1 se cambió el nombre de Puente por Movimiento, para diferenciar más el puente del problema original con el objeto Puente

\item Se mejoraron los textos de Resolución del problema, Algoritmos y Complejidad.

\item Se agregó el algoritmo $enviarArq$ y se expandió el pseudocódigo de $prueboEnviar2Arqueologos$ para mayor claridad.

\item Se explicó el cambio en la complejidad de algunos algoritmos, que en el cálculo se tiene en cuenta la recursión pero para otras operaciones no.

\item Se dio una mejor explicación de porqué el programa termina y una demostración de porqué el algoritmo recorre todos los estados válidos.

\item Se explicó porqué la altura máxima del árbol es de $\mathcal{O}((N+M)!)$

\item Se agregó la descripción del generador de casos aleatorios para graficar los tiempos del algoritmo. Además, se agregó un gráfico fijando N en un caso y M en el otro.

\end{itemize}

\subsection{Problema 2}

\begin{itemize}

    \item Se corrigió la demostración de la complejidad.
    
    \item Se explicó de mejor manera los resultados obtenidos en la experimentación realizada.

\end{itemize}

\subsection{Problema 3}

\begin{itemize}
\item Se explicó de mejor manera función recursiva
\item Se cambió el paso inductivo de la demostración de correctitud para que sea más entendible.
\end{itemize}